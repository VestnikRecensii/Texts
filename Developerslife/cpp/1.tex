\documentclass{article}

\usepackage [utf8] {inputenc}
\usepackage [T2A] {fontenc}
\usepackage[english, russian]{babel}
\usepackage[left=2cm,right=2cm, top=2cm,bottom=2cm,bindingoffset=0cm]{geometry} % page settings
\usepackage[usenames]{color}
\usepackage{colortbl}
\usepackage{xcolor}
\usepackage{framed}
\usepackage{caption,subcaption}
\usepackage{listings}
\usepackage{titling}
\usepackage{soulpos}
\usepackage{soul}

%TODO: use minted package to highlight source code

\definecolor{lightcyan}{rgb}{0.88, 1.0, 1.0}

\lstset{
	extendedchars=true,
	escapeinside={\%*}{*)}, %Escape bad characters inside listings
}

\lstdefinestyle{shared}
{
    numbers=left,
    numbersep=1em,
    frame=single,
    framesep=\fboxsep,
    framerule=\fboxrule,
    rulecolor=\color{red},
    xleftmargin=\dimexpr\fboxsep+\fboxrule\relax,
    xrightmargin=\dimexpr\fboxsep+\fboxrule\relax,
    breaklines=true,
    tabsize=2,
    columns=flexible,
}

\lstdefinestyle{cpp}
{
    style=shared,
    language={C++},
    %alsolanguage={[Sharp]C},
    basicstyle=\small\tt,
    keywordstyle=\color{blue},
    commentstyle=\color[rgb]{0.13,0.54,0.13},
    backgroundcolor=\color{lightcyan},
    stringstyle=\color{orange},
    showstringspaces=false,
    basicstyle=\fontsize{12}{14}\selectfont\ttfamily
}

\lstnewenvironment{cpp}
{\lstset{style=cpp}}
{}

\definecolor{shadecolor}{RGB}{225,241,250}
\title{Введение в C++}
\author{Ярослав Зотов}
\date{Февраль 2018}

\newcommand{\fordummies}[1]{
    \begin{snugshade*}
        #1
    \end{snugshade*}
} %Text format for dumiies

\newcommand{\cppword}[1]{
{\sethlcolor{lightcyan}\hl{#1}}
}

\newcommand{\chaptertitle}[1]{
	\setlength{\voffset}{-7ex}
	\title{\textbf{\Huge #1}}
	\author{\vspace{-1ex}}
	\date{\vspace{-7ex}}
}


\begin{document}
\fontsize{14pt}{16pt}\selectfont  %set font size

\title{\textbf{\Huge Составные части программы}}
\author{}
\date{}

\maketitle 

\section*{Комментарии}

В прошлой главе после написания первой программы я долго-долго расписывал, что значит каждая строчка. Если программа пишется в течение долгого времени несколькими людьми, содержит неочевидные фрагменты, то в ней просто необходимы пометки. Для того, чтобы оставить пометки непосредственно в программе используются комментарии.

\fordummies{

Комментарий — пояснение к исходному тексту программы, находящееся непосредственно внутри кода. Компилятор при работе игнорирует комментарии, поэтому в них может находиться любой текст, включая русские буквы. На то он и комментарий.

}

В C++ комментарии бывают двух видов --- однострочные \cppword{//} и многострочные \cppword{/**/}. Однострочный превращает в комментарий текст от двойного слэша \cppword{//} до конца строки, а многострочный --- весь текст между символами \cppword{/*} и \cppword{*/}. Вот как выглядит первая программа с комментариями:

\begin{lstlisting}[caption={Программа с комментариями}, captionpos=b, style=cpp]
#include <iostream> //%*Подключение заголовка iostream.*)
using namespace std; //%*подключение стандартного пространства имен*)
/*
int main() - %*главная функция.*)
%*Выполняется при старте программы, возвращает результат операционной системе.*)
*/
int main() {
	cout << "Hello, World!" << endl;
	//cout << "2*2 = 5" << endl;
	//%*Предыдущая строка печатала бы неправильный текст, закомментируем ее.*)
	return /*%*Возвращаемое значение*)*/ 0;
}*)
\end{lstlisting}

Сразу стало понятнее, верно?

Комментарии можно ставить между двумя лексемами и там, где могут находиться пробелы или концы строк. Можно быстро выключать фрагменты кода, комментируя их.  Но разрывать слово комментарием нельзя.

Комментарии --- это та часть программы, которая предназначена только для людей, поэтому они должны быть понятны. Их цель --- сделать текст программы проще для восприятия как вам самим, так и другим читателям. Однако, быть Капитаном Очевидность тоже не стоит. Заметьте, есть разница между

%\lstinputlisting[language=Python, firstline=37, lastline=45]{source_filename.py}

\begin{lstlisting}[caption={Абсолютно бессмысленный комментарий}, captionpos=b, style=cpp]
	n = 10; //%*Присвоим n значение 10 *)
\end{lstlisting}

и

\begin{lstlisting}[caption={Feel the difference}, captionpos=b, style=cpp]
	n = 10; //%*10 - значение шага по умолчанию для нашего алгоритма*)
\end{lstlisting}

Но даже с полезными комментариями стоит быть начеку. Иногда программисты меняют код, а про комментарии забывают. Тогда ситация становится еще хуже. В следующем примере мало того, что комментарий противоречит коду, так еще и непонятно, какое значение правильное.

\begin{lstlisting}[caption={Oops}, captionpos=b, style=cpp]
	n = 20; //%*10 - значение шага по умолчанию для нашего алгоритма*)
\end{lstlisting}

Будьте внимательнее, грамотно комментируйте сложные фрагменты кода, не забывайте своевременно обновлять комментарии, и работать с текстом вашей программы будет одно удовольствие.

\section*{Отступы}

\end{document}
