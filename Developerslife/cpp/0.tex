\documentclass{article}
\usepackage [utf8x] {inputenc}
\usepackage [T2A] {fontenc}
\usepackage[english, russian]{babel}
\usepackage[left=2cm,right=2cm, top=2cm,bottom=2cm,bindingoffset=0cm]{geometry} % page settings
\usepackage[usenames]{color}
\usepackage{colortbl}
\usepackage{xcolor}
\usepackage{framed}
\usepackage{caption,subcaption}
\usepackage{listings}

\lstdefinestyle{shared}
{
    numbers=left,
    numbersep=1em,
    frame=single,
    framesep=\fboxsep,
    framerule=\fboxrule,
    rulecolor=\color{red},
    xleftmargin=\dimexpr\fboxsep+\fboxrule\relax,
    xrightmargin=\dimexpr\fboxsep+\fboxrule\relax,
    breaklines=true,
    tabsize=2,
    columns=flexible,
}

\lstdefinestyle{cpp}
{
    style=shared,
    language={C++},
    %alsolanguage={[Sharp]C},
    basicstyle=\small\tt,
    keywordstyle=\color{blue},
    commentstyle=\color[rgb]{0.13,0.54,0.13},
    backgroundcolor=\color{cyan!10},
    stringstyle=\color{orange},
    showstringspaces=false,
    basicstyle=\fontsize{12}{14}\selectfont\ttfamily
}

\lstnewenvironment{cpp}
{\lstset{style=cpp}}
{}

\definecolor{shadecolor}{RGB}{225,241,250}
\begin{document}
\fontsize{14}{16pt}\selectfont  %set font size
\title{Введение в C++}
\author{Ярослав Зотов}
\date{Февраль 2018}

\newcommand{\fordummies}[1]{
\begin{snugshade*}
#1
\end{snugshade*}
} %Text format for dumiies

% \maketitle %uncomment for title

\section{Предисловие}

Недавно ко мне обратился один человек с просьбой сделать ему простую программу --- берутся одни данные и по простейшей формуле из них строятся другие. Тот человек думал, что раз \glqqяжпрограммист\grqq, то для меня решение его задачи не составит труда. Я и вправду справился, но грустная правда  заключалась в том, что для решения вполне хватило стандартных формул Excel и пары-тройки минут свободного времени.

Программированию может научиться любой. Для этого не нужно ни специальное образование, ни специальные двухнедельные курсы, ни особый склад ума. Но, к сожалению, до сих пор часто считается, что программирование сродни магии, которую хорошо бы отдать на откуп волшебникам-программистам, а самим об этом забыть. Но только это так не работает. 

Автоматизировать можно почти все - от офисной работы до квестов. Знание всего лишь основ программирования выделяет человека на рынке труда и сильно упрощает ему жизнь на работе (главное - не соглашайтесь чинить чайники). Умение программировать позволяет взглянуть на повседневные задачи под несколько иным углом. Я считаю, что программирование полезно для всех, каждому, и пусть никто не уйдет обиженным.

Этот материал как раз и предназначен для того, чтобы программирование стало ближе для каждого. Чтобы сделать его еще более простым, я максимально снизил порог вхождения, из-за чего придется объяснять вещи, кажущиеся некоторым очевидными. Они оформлены вот таким образом:

\fordummies{Полковник Фридрих Краус фон Циллергут (Циллергут --- название деревушки в Зальцбурге, которую предки полковника пропили еще в восемнадцатом столетии) был редкостный болван. Рассказывая о самых обыденных вещах, он всегда спрашивал, все ли его хорошо поняли, хотя дело шло о примитивнейших понятиях, например: <<Вот это, господа, окно. Да вы знаете, что такое окно?>> Или: <<Дорога, по обеим сторонам которой тянутся канавы, называется шоссе. Да-с, господа. Знаете ли вы, что такое канава? Канава — это выкопанное значительным числом рабочих углубление. Да-с. Копают канавы при помощи кирок. Известно ли вам, что такое кирка?>> Он страдал манией все объяснять и делал это с воодушевлением, с каким изобретатель рассказывает о своем изобретении. <<Книга, господа, это множество нарезанных в четвертку листов бумаги разного формата, напечатанных и собранных вместе, переплетенных и склеенных клейстером. Да-с. Знаете ли вы, господа, что такое клейстер? Клейстер — это клей>>. 

\begin{flushright}Ярослав Гашек, <<Похождения бравого солдата Швейка>>\end{flushright}}

Если то, что там написано, кажется вам очевидным или слишком упрощенным, смело пропускайте, эта часть материала предназначена для менее посвященных.

Фрагменты исходного кода оформлены следующим образом:

\begin{lstlisting}[caption={Пример оформления кода}, captionpos=b, style=cpp]
#include <iostream>
using namespace std;

int main() {
	cout << "Hello, World!" << endl;
	return 0;
}
\end{lstlisting}

Как вы уже догадались, мы будем изучать программирование на примере C++. Почему не Ruby, Python, Go или что-то другое стильное модное молодежное? И порог вхождения в плюсы довольно высок, и синтаксис далек от совершенства, казалось бы, C++ явно не является языком, с которого надо начинать изучение программирования. Однако, с этим можно поспорить. Си-подобный синтаксис используется ныне в Java, JavaScript, C\#, Objective C и многих других. Некоторая сложность синтаксиса компенсируется широтой его применения. Выучил синтаксис однажды --- используешь постоянно.

C++ за десятилетия своего развития объединил под своим крылом самые разные концепции и возможности --- здесь можно вручную управлять памятью, а можно положиться на автоматические контейнеры, можно использовать средства из восьмидесятых годов в разработке современного мобильного приложения, можно писать максимально быстрый и близкий к низкому уровню код и в то же время защищать пользователя с помощью исключений.

C++ ныне имеет уже солидный возраст, но все еще успешно борется с более молодыми --- на нем пишут компиляторы и другие языки, браузеры, игры, среды разработки, мобильные и десктопные приложения и многое другое. Хотите быть с теми, кто этим занимается? Вливайтесь.

\section{История вопроса}

Давным-давно компьютеры были большими, а программы маленькими. Они представляли собой машинные команды в виде строк нулей и единиц (многие люди еще с тех пор считают смешными шутки про двоичный код и программистов). Несколько позже был создан ассемблер, где каждая команда представляет собой слово или сокращение на английском языке. А с появлением языков высокого уровня жизнь программистов стала почти сказочной --- теперь для написания программы использовался какой-никакой язык с конструкциями из слов и предложений. Можно было написать что-то вроде <<Пусть а равно 1>> и компилятор или интерпретатор преобразовывал эту запись в понятный машине код.

\fordummies{

Компилятор --- программа, переводящая текст с языка высокого уровня на язык низкого уровня, близкий машинному коду. Например, на входе мы имеем текст программы на C++, а на выходе --- исполняемый exe-файл.

Интерпретатор же поступает иначе - пользователь скармливает ему текст программы, а тот построчно его анализирует и выполняет. Например, так работает Python, Ruby или JavaScript под капотом веб-страницы.

Т.о. программа на компилируемом языке представляет собой две сущности --- исходный код до компиляции и двоичный файл после, а у интерпретируемых языков никакой разницы нет, там исходный код является одновременно и текстом программы и самим исполняемым файлом. Программа на интерпретируемом языке сможет работать на любой машине, где есть соответствующий интерпретатор, однако без интерпретатора она бесполезна. А скомпилированная программа будет работать только на соответствующей операционной системе и архитектуре, но для ее запуска дополнительно программное обеспечение вроде компилятора уже не нужно (хотя могут быть нужны различные библиотеки, но пока не стоит вдаваться в такие подробности).

}

Программисты продолжали совершенствовать свои рабочие инструменты и создавали новые, и в начале 1980-х годов свет увидело детище Бьёрна Страуструпа --- язык, первоначально названный Си с классами. 

Работая в компании Bell Labs, Страуструп сталкивался с задачами, решать которые уже существующими средствами было неудобно --- одни языки были выразительны, высокоуровневы и удобны, но им не хватало скорости, другие слишком походили на ассемблер, а разрабатывать на нем большие программные продукты --- сомнительное удовольствие. Бьёрн Страуструп (тогда ему было около тридцати) вспомнил молодость и свою диссертацию и дополнил язык Си высокоуровневыми возможностями, добавив классы, наследование, строгую проверку типов, inline-функции и аргументы по умолчанию (не переживайте, если часть этих слов вам незнакома, со временем мы до них доберемся).

Страуструп написал транслятор, переводящий исходный код C с классами в исходный код чистого C. Благодаря такой совместимости новый язык быстро набрал популярность , но вскоре он перестал быть просто дополнением С и был переименован в C++. В 1985 году состоялся первый коммерческий выпуск, а вот первый международный стандарт языка появился спустя аж 13 лет --- в 1998 году. Следующие версии стандарта вышли в свет в 2003, 2011 и 2014 годах.

За годы своего развития C++ получил множество новых возможностей, потерял совместимость с C, а также приобрел множество яростных сторонников и противников. Бьёрн Страуструп как-то сказал: <<Есть всего два типа языков программирования: те, на которые люди всё время ругаются, и те, которые никто не использует>>. Судя по всему, C++ еще долго будет принадлежать к первому типу.

\section{Собственно, C++}

Для работы нам понадобятся компилятор с текстовым редактором или среда разработки. Сейчас их огромное множество и каждый может выбрать себе по вкусу. Если вы затрудняетесь с выбором, просто откройте сайт поисковой системы и введите там одну из фраз:

\begin{itemize}
\itemкомпилятор C++ для Windows/Linux/OS X
\itemтекстовый редактор для программиста
\itemсреда разработки C++
\end{itemize}

Несколько минут поиска и вы найдете то, что вам нужно. А теперь перейдем непосредственно к языку.

C++, так же как и любой другой естественный или искусственный язык, состоит из символов. Значащими, т.е. теми, которые имеют в коде значение, являются 26 строчных и 26 заглавных английских букв, подчеркивание \_, все десятичные цифры, а также 29 специальных символов: 

 ! "\#\%\&'()*+,-./:;<=>?[\textbackslash]\textasciicircum\{|\}\textasciitilde

Остальные символы являются незначащими, например, кириллица или китайские иероглифы.

Из значащих символов составляют лексемы (единицы языка со смыслом, аналог слов в естественных языках)  --- служебные слова, идентификаторы, знаки операций, разделители и литералы.

\begin{lstlisting}[caption={Пример лексем}, captionpos=b, style=cpp]
int i  = 0;
\end{lstlisting}

В примере выше int --- служебное слово, целый тип, i --- идентификатор, имя переменной, пробелы до и после знака равно --- разделители, сам знак = --- знак операции присваивания,  0 --- литерал (фиксированное значение), число ноль.

\fordummies{

Служебное слово --- слово, имеющее в языке программирования специальное значение. Например, const означает некоторое константное, неизменяемое выражение.

Идентификатор --- уникальный признак объекта, позволяющий отличать его от других объектов, или, попросту имя. С их помощью именуют переменные, функции, метки и т.п.

Знак операции --- аналог математических сокращений для операций. Например, i = 0 означает присвоить сущности под именем i значение 0.

Разделитель --- все, что отделяет одну лексемуот другой --- пробелы, символы табулции, концы строк и комментарии.

Литерал --- фиксированное значение, например, число, строка, логическое значение true или false и т.п.

}

\section{Наша первая программа}

В программировании принято начинать обучение новому языку с написания простейшей программы, которая выводит на экран фразу <<Hello, world!>>. Так эта программа выглядит на C++:

\begin{lstlisting}[caption={Наша первая программа <<Hello, world!>>}, captionpos=b, style=cpp]
#include <iostream>
using namespace std;

int main() {
	cout << "Hello, World!" << endl;
	return 0;
}
\end{lstlisting}

\section{Итог}

\end{document}
